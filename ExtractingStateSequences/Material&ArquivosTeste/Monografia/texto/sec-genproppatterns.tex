\section{Generation of properties from most frequent test case patterns}
\label{genproppatterns}

Based on the patterns obtained through the usage of $PrefixSpan$ shown previously, we can automatically define formal properties to be verified. For this project we chose two property specification patterns, explained in Section \ref{specpatterns}, from \cite{dwyer98}: the response and existence patterns. First, we go through the most frequent patterns based on a user minimum support and compute the formal response properties. Second, we  try to find patterns that are common to all of the test cases to be able to apply the existence specification.

\subsection{Response property specification}
\label{responseSpec}

The creation of response properties is done using as input the patterns mined by $PrefixSpan$ in the previous step. In each sequence pattern, we obtain pairs of consecutive events and establish a property that the second event must respond to the first event of the pair. The global scope was used. The pseudocode can be found below:

\begin{lstlisting}[mathescape,caption={Pseudocode to extract response properties from the mining results}]
//Method to write the specification of formal response properties
//It receives as argument a sequence pattern 

Set propertySet = new Set();

Set extractResponseProperties(Sequence pattern) {

	for (i = 0; i < pattern.length - 1; i++) {
		Event P = pattern.getEvent(i);
		Event S = pattern.getEvent(i+1);

		Property responseProperty = $\Box$(P.getName() $\rightarrow$ $\Diamond$ S.getName());

		if (!propertySet.contains(responseProperty))
			propertySet.add(responseProperty);
	}

	return propertySet;
}
\end{lstlisting}

To illustrate this phase, take as example the test cases automatically generated for statechart in Figure\ref{fig:trocaPlano}, presented in Table \ref{testPathsTrocaPlano}. With a minimum support of $70\%$, the most frequent patterns returned are displayed in Table \ref{patternsTrocaPlano}. Applying each one of these patterns as input to the method \textit{extractResponseProperties}, we obtain the following response properties:

\begin{table}[h]
\begin{center}
\begin{tabular}{|p{7cm}| l|}

\hline

Informal description & Formal specification \\ \hline

searchLineNumber responds to doLogin & $\Box (doLogin \rightarrow \Diamond searchLineNumber)$ \\ \hline

searchEmployee responds to doLogin & $\Box (doLogin \rightarrow \Diamond searchEmployee)$ \\ \hline

searchEmployee responds to searchLineNumber & $\Box (searchLineNumber \rightarrow \Diamond searchEmployee)$ \\

\hline
\end{tabular}
\end{center}
\caption{Properties automatically extracted from patterns in \ref{patternsTrocaPlano}.}
\label{propertiesTrocaPlano}
\end{table}

These properties reflect some of the requirements of the specification. The login has to be performed so that the system can access users information: the corresponding users line number or if users are employees of the company, for example. Besides that, according to the flow described by the statechart (Figure \ref{fig:trocaPlano}), the employee verification should be done after the line number is retrieved, as stated in the third property. 

It is possible to realise that some of the generated properties contain events in common. Therefore, we can combine them in order to reduce the number of properties that should be verified by the model checker. Let's consider, for example, the first and second properties described in Table \ref{propertiesTrocaPlano}: 

\begin{center}

$\Box (doLogin \rightarrow \Diamond searchLineNumber)$

$\Box (doLogin \rightarrow \Diamond searchEmployee)$

\end{center}

Events \textit{searchLineNumber} and \textit{searchEmployee} respond to the same event \textit{doLogin}. Hence, we synthesise both properties in a new more concise one:

\begin{center}
$\Box (doLogin \rightarrow \Diamond (searchLineNumber \wedge searchEmployee))$
\end{center}

Then, we are left only with to properties to be passed to the model checker:

\begin{table}[h]
\begin{center}
\begin{tabular}{|p{5cm}| l|}

\hline

Informal description & Formal specification \\ \hline

searchLineNumber and searchEmployee responds to doLogin & $\Box (doLogin \rightarrow \Diamond (searchLineNumber \wedge searchEmployee))$\\ \hline

searchEmployee responds to searchLineNumber & $\Box (searchLineNumber \rightarrow \Diamond searchEmployee)$ \\

\hline
\end{tabular}
\end{center}
\caption{Combined properties from \ref{propertiesTrocaPlano}.}
\label{propertiesTrocaPlanoConcise}
\end{table}

\subsection{Existence property specification}

In order to use the existence specification pattern, we must find sequence patterns that are present in the whole set of test cases. In other words, we should perform the test case mining stage with a minimum support of $100\%$. If any such pattern is found, we use the global scope and define an existence property, meaning that for every path taken, we will eventually find that pattern. Considering that a list of events was found as patterns with support of $100\%$, the following pseudocode can be used:

\begin{lstlisting}[mathescape,caption={Pseudocode to extract existence properties from the mining results}]
//Method to write the specification of formal existence properties
//It receives as argument a list of events that  were found during mining with support of $100\%$

Set propertySet = new Set();

Set extractExistenceProperties(List commonEvents) {

	for (i = 0; i < pattern.length; i++) {
		Event P = pattern.getEvent(i);

		Property existenceProperty = $\Diamond$(P.getName());

		if (!propertySet.contains(existenceProperty))
			propertySet.add(existenceProperty);
	}

	return propertySet;
}
\end{lstlisting}

Still considering test cases in Section \ref{testPathsTrocaPlano} to illustrate the process, the only event returned by the mining phase with $100\%$ of support is \textit{doLogin}. Thus, we are able to specify one existence property that indicates that the event \textit{doLogin} must occur, no matter the execution path is taken:

\begin{table}[h]
\begin{center}
\begin{tabular}{|l | l|}

\hline

Informal description & Formal specification \\ \hline

doLogin must occur & $\Diamond (doLogin)$ \\

\hline
\end{tabular}
\end{center}
\caption{Existence property extracted from \ref{testPathsTrocaPlano}.}
\label{existencePropertyTrocaPlano}
\end{table}

In the that case more than one event is present in all test cases, then we can combine them to create a single existence property to be verified by the model checker. Suppose events $a, b$ and $c$ are present in all test cases of a certain statechart model. Then, we could combine their existence properties in the following single property:

\begin{center}
$\Diamond (a \wedge b \wedge c)$
\end{center}
