%% ------------------------------------------------------------------------- %%
\section{Statecharts}
\label{sec:statecharts}

A especifica��o de um software � um documento de refer�ncia que cont�m os requisitos que o programa
deve satisfazer. Ela pode ser elaborada em linguagem natural, com casos de uso por exemplo, ou utilizando
t�cnicas de engenharia formal de software.

Statecharts s�o um tipo de especifi��o formal de software baseado em diagramas de estado.
Eles podem ser considerados m�quinas de estados finitas (MEF) extendidas, pois, assim como um aut�mato, um statechart possui um conjunto de estados e transi��es entre eles. Por�m, o statechart pode fazer uso das seguintes propriedades:

\begin{itemize}
\item Ortogonalidade

Um statechart pode estar em mais de um estado simultaneamente, o que � �til para modelar situa��es de concorr�ncia e paralelismo.
\item Hierarquia

Um estado pode conter sub-estados e transi��es internas, elevando a capacidade de abstra��o e encapsulamento.
\item Condi��es de guarda

� poss�vel condicionar a ocorr�ncia de uma transi��o baseando-se no valor de uma vari�vel, entrada de um estado ou ocorr�ncia de uma a��o
\item Hist�ria
\end{itemize}

O que

Diferencas de um statechart para um automato
	Ortogonalidade
	Hierarquia
	Condicoes de guarda
	Historia	

Pra que serve/onde eh usado

Exemplo
