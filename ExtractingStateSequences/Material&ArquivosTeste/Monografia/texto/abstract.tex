\chapter*{Abstract}

This project explores the theme of software quality through the perspectives of testing and formal methods. Testing is a crucial activity in the software development cycle, since it increases the reliability of the system. Generally, the test cases executed are created based on the requirements contained in the specification. Formal methods also aims the quality assurance and provides several techniques and tools to be used during specification, design and verification phases. In particular, statecharts are a kind of formal specification based on finite state machines mostly used to model reactive system behaviours. Testing and formal methods should be seen as complementary approaches, since testing can only show the presence of errors, but formal verification, such as model checking, can prove their absence. However, in order to use formal verification, one needs to define properties in a certain specification language, which requires strong mathematical background. In order to facilitate the creation of test cases and the specification of formal properties by developers, two techniques were studied and implemented: First, we automatically generate test cases for a given Statechart model. Second, we automatically synthesize properties based on the test cases previously obtained. This monograph was prepared for the course MAC0499 - Final Graduation Project, at the Institute of Mathematics and Statistics of University of S�o Paulo. \\
\noindent \textbf{Keywords:} Statecharts, Testing, Test cases, Formal properties

