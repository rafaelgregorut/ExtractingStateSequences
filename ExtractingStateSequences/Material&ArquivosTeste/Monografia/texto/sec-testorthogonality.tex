\section{Test cases for complex statecharts: orthogonality}
\label{testOrthogonality}

Now, we shall consider statecharts that comprise orthogonality (states in concurrent regions).

A method to generate test cases dealing with orthogonality is to eliminate it by flattening the statechart, as explained in \ref{flattening}. The elimination of orthogonality would be done with the Cartesian product of all states and transitions, causing an explosion in the number of result states and transitions \cite{bogdanov}.

To avoid states and transitions explosion and still be able to cover all states and test all transitions, \cite{bogdanov} offers an alternative approach to refine the concurrency requisites. In the present project, we chose to use the strong concurrency refinement:

\begin{itemize}

\item \textbf{Strong concurrency}

This refinement allows us to test concurrent components separately. Transitions from each concurrent region are triggered one-by-one in different steps. In this case, we consider that the concurrent regions are placed in units which either run in parallel or in different processors. So no concurrent region may cause missing transitions in another one or misdirected transitions.
\end{itemize}

The communication resources, as broadcasting, should be disabled during testing since it could cause series of transitions to occur. A chain reaction would be an example of such series and would not be expected by the test cases listed using this implementation.

In order to create the test cases for statecharts that comprise concurrent regions,  we first compute the coverage paths for each concurrent region separately. Then, similarly to the case with hierarchy, we combine these paths with the coverage path of the state that contains the concurrent regions. After obtaining the coverage path for all states, set $C$ is complete. 

In the pseudocode below, we consider that substates are in a region inside the superstate. To apply it to statecharts with hierarchy, discussed in section \ref{testHierarchy}, we must consider that the superstate have only one internal region, which will contain the substates. In statecharts with concurrency, the concurrent elements must be in different regions inside a superstate (Figure \ref{fig:stockUpdateEmail} serves as an example to illustrate all this). The solution presented in the pseudocode can be applied to states with orthogonality as well as to ones with hierarchy only. It first calculates the \textit{State Cover} set, set $C$:

\begin{lstlisting}[mathescape,label={testOrthogonalityPseudo},caption={Recursive State Cover construction for a statechart with orthogonality}]
//Recursive function that will do all the work
//returns the State Cover set, or set C
Set constructSetCRec(State s, Path p, Set setC, List visited) {

	visited.add(s);

	setC.add(s,p);

	if (s.containsSubRegions()) {

		for (Region r in s.getSubregions()) {
			
			Set subSetC = constructSetC(r.getInitialSubstate());	

			r.addSubpaths(subSetC);

			setC.remove(s,p);

			for (State substate in s.getSubstates()) {
	
				Path partialPath = subSetC.getPath(substate);
	
				Path substateCoveragePath = p + partialPath;
	
				setC.add(substate,substateCoveragePath)	
			}
		}	
		
	}
	
	for (Transition t in s.getOutgGoingTransitions()) {
		
		State nextState = t.getDestiny();

		if (!visited.contains(nextState)) {

			if (s.containsSubstates()) {
			
				Path nextCoveragePath = p + $\Delta_{s}$ + t.getLabel());

			} else {

				Path nextCoveragePath = p + t.getLabel());

			}	

			constructSetCRec(nextState,nextCoveragePath,setC,visited);	
		}
	}
	
	return setC;
}
\end{lstlisting}

Once the $C$ set is built, the test cases for each transition of the model can be generated. This process is the same as the one described in the case with hierarchy in \ref{testHierarchy}, since concurrent states are inside a superstate. %The algorithm for this phase can be found in pseudocode format in \ref{pseudocodeTestCase}.
%\todo[inline]{para manter uma numera��o para os c�digos � preciso colocar um caption em cada lisitng}


\begin{figure}[htb]
\centering
\includegraphics[width=15cm]{figuras/stockUpdateEmail}
\caption{\label{fig:stockUpdateEmail} Statechart model for concurrent jobs: clear reservations job and send email job}
\end{figure}

To illustrate this case, we can look at the example provided in Figure \ref{fig:stockUpdateEmail}. It models an e-commerce application in which the administrator is allowed to execute two jobs: one is to clear all the reservations of products (region \textit{Clear reservations job} in the figure) and the other one is to send emails to customers letting them know certain products are back in stock (region \textit{Email job} in the figure). Both jobs can run in parallel if the administrator wishes, thus their regions are modelled in a concurrent way in the statechart. This statechart models the requirement described in Subsection \ref{req4}.

According to our refinement, the coverage paths will be created for substates in \textit{Clear reservations job} and \textit{Email job} separately. First, the substates in \textit{Clear reservations job} concurrent region are inside state \textit{Stock update email}, hence we need to apply the algorithm presented above. 
%in \ref{testOrthogonalityPseudo}. 
Note that since the coverage path of \textit{Stock update email} is just the empty string $e$, it will not have a great impact on the substates paths. Let's consider \textit{Product retrieved} and \textit{Stock level updated} to exemplify the results:

\begin{center}
\begin{tabular}{| p{4cm} | p{10cm}|}

\hline

State & Coverage path \\ \hline

Product retrieved & e e startClearJob retrieveReservedProducts nextProduct \\ \hline

Stock level updated & e e startClearJob retrieveReservedProducts nextProduct updateStockLevel \\

\hline
\end{tabular}
\end{center}

The coverage paths for substates in region \textit{Email job} can be built using a similar process. For instance, the coverage path for \textit{Email sent} is:

\begin{center}
\begin{tabular}{| p{4cm} | p{10cm}|}

\hline

State & Coverage path \\ \hline

Email sent & e e  startEmailJob filterStockUpdates retrieveUsersEmails sendEmails\\ 
\hline
\end{tabular}
\end{center}

The generation of the test cases, then, is similar to the one presented in Section \ref{testHierarchy}. But, when there is orthogonality, we need to consider the coverage paths of substates from all concurrent regions in a state during the expansion phase. For the example in Figure \ref{fig:stockUpdateEmail}, since there is no state after \textit{Stock updated email}, no expansion of coverage paths will be needed.

For example, let's consider the test cases for transitions \textit{nextProduct}, from state \textit{Stock level updated}, and \textit{sleep}, from state \textit{Email sent}. We need to append these transition labels to the coverage path of their origin state. Therefore, \textit{nextProduct} will be appended to the coverage path of \textit{Stock level updated}, and \textit{sleep} to the coverage path of \textit{Email sent}:

\begin{itemize}

\item Test case for \textit{\textbf{nextProduct}}

Origin state: \textit{Stock level updated}

Test Path: \textit{e e startClearJob retrieveReservedProducts nextProduct updateStockLevel nextProduct}

Expected state: \textit{Product retrieved}

\item Test case for \textit{\textbf{sleep}}

Origin state: \textit{Emails sent}

Test Path: \textit{e e  startEmailJob filterStockUpdates retrieveUsersEmails sendEmails sleep}

Expected state: \textit{Idle}

\end{itemize}
