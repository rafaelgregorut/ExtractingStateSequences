\chapter{Test cases}
\label{cap:testgen}

In this section, we will be concerned with test case creation from the functional perspective. The structure of the code will not be analyzed. Instead, the specifications with the requisites are going to be the source to derive the test cases.

\input sec-testcasedesign

\input sec-autotestcases

\section{Implementation test case generation for statecharts}

In this project, we implemented the test case generation for statecharts based on the criteria described in \cite{bogdanov}. We test every transition by visiting every state and trigger events for all transitions that start in it. 

\subsection{Test case for simple statecharts}

For this section, we consider only statecharts that do not contain hierarchy and concurrency. Statecharts with hierarchy and concurrency will be explained later.


We start by making sure every reachable state in the statechart is covered. In order to do so, for each state $s$ in the statechart, we construct a path $p$ from the initial state to $s$. The path $p$ in said to be the coverage path of $s$. All coverage paths generated are stored in a set called $State Cover$, which is denoted by $C$. Therefore, C is a set of sequence of transition labels, such that we can find an element from this set to reach any desired state starting from the initial one \cite{bogdanov}.

Since there is no hierarchy or concurrency in the statechart, the construction of $C$ is similar to covering states of an automaton. The process can be done through a depth search:

@@@@@ ADICIONAR PSEUDO-CODIGO DA COBERTURA

Now that we have the coverage for every reachable state, we need to trigger each transition on each state and create the test cases. For each transition there will be a test case, thus every transition in the diagram will be exercides at least once during testing.

Consider the transition $t = (s,e,q)$, where $s$ is the original state, $e$ is the event label that triggers $t$ and $q$ is the destinity state. Previously, we computed that $s$ has coverage path $p$ such that $p \in C$ and $p$ is a sequence of label. The test case $TC$ to $t$ would be the concatenation of the event label $e$ to the end of $p$ expecting to get to state $q$. The process is repeated to every transition in the statechart.

@@@@@ADICIONAR EXEMPLO!!!!!

\subsection{Test case for complex statecharts: hierarchy}

\subsection{Test case for complex statecharts: orthogonality}
