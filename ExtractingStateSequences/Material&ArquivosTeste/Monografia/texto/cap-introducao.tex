%% ------------------------------------------------------------------------- %%
\chapter{Introduction}
\label{cap:introducao}

bla bla bla

\section{Motivation}

In the model driven development, test cases to validate systems can be generated from a model that represents how the system must behave. For instance, if the services to be provided by a system are described in Statecharts models, one can  automatically generate test cases for programs from such models. However, properties to be proved in a system are not synthesized from such models. In general, developers take the system specification and observe its restrictions to then define which properties should be satisfied. 


This project proposes a study on the set of test cases generated from specifications in order to guide properties definitions. First, a technique to automatically generate test cases from the specification is presented. Then, all test cases are observed and a technique to synthesize properties from the acquired tests is proposed.

\section{Goals}

\begin{itemize}

\item Implement an algorithm to automatically generate test cases based on statechart specifications

\item Implement a technique to automatically analyse a set of test cases and extract the events, and sequences of events, that are found in such whole set

\item Using the previous information, synthesize properties to be used by model check programs 

\item Develop a tool to help these tasks

\end{itemize}

\section{Organization}

In chapter \ref{cap:conceitos}, we present the concepts studied to make this project possible. In chapter \ref{cap:testgen}, we describe an technique implemented to automatically generate test cases from statechart models. In chapter \ref{cap:propextract}, we propose a technique to synthesize formal properties based on the previously obtained test cases. Our conclusions regarding this project can be found in chapter \ref{cap:conclusoes}. Information about the tool used to create the statechart models are in apendix \ref{ape-yakindu}. Finally, the syntax of the linear temporal logic, the logic used in the generated formal properties, can be found in apedix \ref{ape-ltl}.
